\section{Research Systems}
This section describes systems that have been developed in the lab but not
commercialized.

\begin{table}[H]
\begin{tabular}{lllllll}
\textbf{Data Rate} & \textbf{Range} & \textbf{Spectrum} & \textbf{Modulation} &
\textbf{RX Type} & \textbf{TX Type} & \textbf{Power} \\
1 Gbps      & 2m     & 532nm & OOK         & APD & Laser & 10mW           \\
2.28 Mbps   & 50m    & 470nm & DPIM        & APD & LED   & 10W            \\
1.2 Mbps    & 30m    & 480nm & DPIM        & APD & LED   & 5W             \\
2.3 Gbps    & 7m     & 520nm & OOK         & APD & Laser & 12mW           \\
1.45 Gbps   & 4.8m   & 450nm & IM/DD OFDM  & APD & Laser & 40mW           \\
4.8 Gbps    & 5.4m   & 450nm & 16-QAM-OFDM & APD & Laser & 15mW           \\
1 Mbps      & 3m     & 532nm & BPSK        & PMT & Laser & 3W             \\
5 Mbps      & 3.6m   & 532nm & 32-QAM      & PMT & Laser &                \\
1 Mbps      & 3.66m  & 405nm & OOK         & PMT & Laser & 17mW           \\
161.35 Mbps & 2m     & 450nm & 16-QAM OFDM & APD & LED  &
\end{tabular}
\end{table}

A few conclusions can be drawn from these results:
\begin{enumerate}
\item Simple IM/DD modulation can achieve good performance.
\item Laser based systems can provide much higher range than \ac{LED} based.
\item High data rates (gigabit) have not been observed past 7 metres.
\item \ac{APD} receivers are capable of receiving high transmission rates.
\item \ac{OFDM} (with multiple links) may be required to achieve further
distances with high data rates, as they allow for diversity.
\end{enumerate}
