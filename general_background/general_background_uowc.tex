\documentclass{article}
\usepackage[colorlinks]{hyperref}
\usepackage[printonlyused,withpage]{acronym}
\author{Christopher Bainbridge}
\title{Underwater Optical Wireless Communications - A Background}

\begin{document}
\maketitle

\begin{abstract}
This document serves as a background and brief summary of \ac{UOWC}. 
\end{abstract}

\section{Introduction}
There are three main types of communicating wirelessly underwater - acoustic, RF and optical. Acoustic transmissions are characterized by long range but low data rates. RF transmissions are characterized by high throughput but only at very short distances. Optical transmissions sit halfway between these, allowing high throughput at medium ranges, but at the expense of being affected by the environment of the wireless channel.

\section{Light Propagation in Water}
This section lists with short descriptions the issues \ac{UOWC} communications face underwater.

\subsection{Absorption and Scattering}
A small fraction of light will be absorbed, and another part scattered, by the water. This becomes more prevalent as the water type degrades, thus propagation is far more challenging.

\subsection{Turbulence}
The refraction index can vary along the propagation path due to fluctuations in the density, salinity and temperature of the underwater environment. This is known as scintillation and will degrade performance.

\subsection{Pointing and Alignment}
Optical beams are very narrow and thus LOS must be maintained for reliable link performance. Tracking is required between nodes to maintain this.

\subsection{Background Noise}
As the visible spectrum of light is being used, noise from other light sources including the Sun can have an effect. In general, deep ocean is less noisy than harbor side.

\subsection{Multipath interference and dispersion}
Multipath interference is produced when an optical signal reaches the detector after encountering multiple scattering objects or reflections from other underwater bodies. The signal can be time dispersed thus decreasing the symbol rate due to \ac{ISI}. It is very dependent on the operating conditions, and has been concluded that this effect is only really present in highly turbid environments. Spatial diversity may help to reduce the effects of multipath interference.

\subsection{Physical obstructions}
Anything that might get in the way such as marine animals will cause momentary loss of signal at the receiver. The light will also attract marine animals, especially at depth. Error correction, signal processing techniques and redundancy measures are required to ensure re-transmission of data when lost.



\section{Water Types}
The literature classifies water types in \ac{UOWC} into these main classes.

\subsection{Pure Water and Pure Sea Water}
Pure water is the cleanest water type and has no suspended particulate matter. Pure sea water is pure water but with the addition of salts, although this is assumed to be negligible in visible spectrum, thus making pure sea water and pure water equivalent.

\subsection{Coastal Ocean}
Coastal ocean displays more severe absorption and scattering due to the increased concentration of dissolved particles.

\subsection{Turbid Harbor}
This is the most hostile environment for optical communications, with the highest concentration of suspended and dissolved particles leading to increased absorption and scattering.




\section{Configurations of UOWC}
There are 4 types of link configurations with regards to \ac{UOWC}.

\subsection{Point to Point \ac{LOS}}
This is the most typical configuration. Both nodes employ the same transceiver and light sources. Typically the beam angles are narrow, thus requiring precise pointing between nodes.

\subsection{Diffused \ac{LOS}}
A point to point link with a large divergence angle allows a signal to be 'broadcast' from one node to multiple remote nodes. Attenuation will be large though due to the increased interaction area with the water. This is useful only if short communication distances and lower data rates are required.

\subsection{Retro-reflector \ac{LOS}}

Instead of having both ends transmit a signal, the remote node reflects back the received light whilst encoding its response on it. This is ideal for nodes with low power and weight requirements, but the signal will degrade significantly due to backscatter and additional attenuation through the water.

\subsection{\ac{NLOS}}
The signal can be 'bounced' off the sea surface, allowing transmissions to propagate further if there are underwater obstacles present. Of course, the light will be reflected in a somewhat random way depending on wind or other turbulence sources, causing signal dispersion and degradation.


\section{System design}
A generic \ac{UOWC} system incorporates a source generating information to be transmitted, a transmission modulator onto the optical carrier, a transmitter equipped with projection optics and beam steering elements to focus and steer the beam, a detector converting optical to electrical signals and then a subsequent signal processing unit and demodulator.

\subsection{Transmitter}

\subsection{Receiver}

\subsection{Modulation Schemes}

\subsection{Coding Schemes}

\section{Research Systems}
This section describes systems that have been developed in the lab but not commercialized.

\section{Current Optical Systems}
\subsection{\ac{UOWC}}
There currently exists two commercial \ac{UOWC} systems.

\subsection{Bluecomm \ac{UOWC}}
This is manufactured by Sonardyne, a UK based company. It utilizes high power \ac{LED}s with a range of products depending on application, from shallow water to deep water, and when there are artificial lights being used (for example on \ac{ROV}s). They claim a maximum transmission rate of 500 Mb/s but for very short distances; longer distances are limited to around 5 - 10 Mb/s.

\subsection{Anglerfish \ac{UOWC}}
The Anglerfish \ac{UOWC} provides full-duplex voice communication between two divers, again using \ac{LED}s. It is designed for the military, and is difficult to intercept because of the directionality and unknown modulation frequencies.

\subsection{Optical Wireless Communication}
As well as underwater systems, there are also terrestrial based commercial systems utilizing visible light.

\subsubsection{Li-Fi}
Li-Fi was coined by Professor Harald Haas. The aim is to run standard 802.11 protocols using visible light (using infrastructure already in place such as overhead bulbs), hence the similarity to Wi-Fi.

There are two companies involved with Li-Fi, but PureLifi is the only one with a system on the market. They offer the PL0300, an integrated device implementing an 802.11 compatible stack with a USB interface. They claim 86.4 Mb/s data rate.

See https://purelifi.com/wp-content/uploads/2019/02/pureLiFi-PL0300-ASICDatasheet.pdf

The other is Velmenni based in India.

\subsubsection{IrDA / IrLAN}
The Infrared Data Association provide a specification for gigabit transmissions using infrared light. IrLAN allows a unit to connect to a local area network. However, it has dropped out of use and there aren't really any available modules on the market anymore.

\subsection{Wired}
Wired interfaces designed for lasers may provide the exact requirements.

However, wired interfaces can really only provide point to point communications. They have no knowledge of multi-point setups. Due to the use of a very directional laser, however, this may not be an issue.

\subsubsection{Koruza}
This is an implementation of \ac{SFP}-based optical transmissions using infra-red light.

KORUZA - https://github.com/IRNAS/KORUZA

There are currently no available \ac{SFP} transceivers working at the visible light (350nm and 420nm).


\section{List of Acronyms}
\begin{acronym}
\acro{AGC}{Automatic Gain Control}
\acro{APD}{Avalanche Photodiode}
\acro{BER}{Bit Error Rate}
\acro{FEC}{Forward Error Correction}
\acro{FOV}{Field Of View}
\acro{IM/DD}{Intensity Modulation / Direct Detection}
\acro{ISI}{Inter-Symbol Interference}
\acro{I2C}{Inter-Integrated Circuit}
\acro{KAUST}{King Abdullah University of Science and Technology}
\acro{LDPC}{Low Density Parity Code}
\acro{LED}{Light Emitting Diode}
\acro{LOS}{Line Of Sight}
\acro{MAC}{Medium Access Controller}
\acro{NLOS}{Non Line Of Sight}
\acro{OFDM}{Orthogonal Frequency Division Multiplexing}
\acro{OOK}{On-Off Keying}
\acro{PCIe}{Peripheral Component Interconnect Express}
\acro{PIN}{P-I-N}
\acro{PMT}{Photo Multiplier Tube}
\acro{PPM}{Pulse Position Modulation}
\acro{PWM}{Pulse Width Modulation
\acro{P2P}{Peer to Peer}
\acro{RF}{Radio Frequency}
\acro{ROV}{Remotely Operated Vehicle}
\acro{RS}{Reed-Solomon}
\acro{RSSI}{Received Signal Strength Indicator}
\acro{SFP}{Small form-factor pluggable}
\acro{SNR}{Signal to Noise Ratio}
\acro{SPAD}{Single Photo Avalanche Diode}
\acro{SPI}{Serial Peripheral Interface}
\acro{UART}{Universal asynchronous receiver/transmitter}
\acro{UOWC}{Underwater Optical Wireless Communication}
\acro{USB}{Universal Serial Bus}
\end{acronym}


\end{document}\grid
