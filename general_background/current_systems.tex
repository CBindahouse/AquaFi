\section{Current Optical Systems}
This section describes the currently commercially available systems, both
underwater and terrestrial based, wireless and wired, but all optical based.

\subsection{\ac{UOWC}}
There currently exists two commercial \ac{UOWC} systems.

\subsubsection{Bluecomm \ac{UOWC}}
This is manufactured by Sonardyne, a UK based company. It utilizes high power
\ac{LED}s with a range of products depending on application, from shallow water
to deep water, and when there are artificial lights being used (for example on
\ac{ROV}s). They claim a maximum transmission rate of 500 Mb/s but for very
short distances; longer distances are limited to around 5 - 10 Mb/s.

\subsubsection{Anglerfish \ac{UOWC}}
The Anglerfish \ac{UOWC} provides full-duplex voice communication between two
divers, again using \ac{LED}s. It is designed for the military, and is
difficult to intercept because of the directionality and unknown modulation
frequencies.

\subsection{Optical Wireless Communication}
As well as underwater systems, there are also terrestrial based commercial
systems utilizing visible light.

\subsubsection{Li-Fi}
Li-Fi was coined by Professor Harald Haas. The aim is to run standard 802.11
protocols using visible light (using infrastructure already in place such as
overhead bulbs), hence the similarity to Wi-Fi.

There are two companies involved with Li-Fi, but PureLifi is the only one with
a system on the market. They offer the PL0300, an integrated device
implementing an 802.11 compatible stack with a USB interface. They claim
86.4 Mb/s data rate.

See \url{https://purelifi.com/wp-content/uploads/2019/02/pureLiFi-PL0300-ASICDatasheet.pdf}

The other is Velmenni based in India, although they don't have any products
on the market.

\subsubsection{IrDA / IrLAN}
The Infrared Data Association provide a specification for gigabit transmissions
using infrared light. IrLAN allows a unit to connect to a local area network.
However, it has dropped out of use and there aren't really any available
modules on the market anymore.

\subsection{Wired}
Wired interfaces designed for lasers may provide the exact requirements.

However, wired interfaces can really only provide point to point
communications. They have no knowledge of multi-point setups. Due to the use
of a very directional laser, however, this may not be an issue.

If in-band localization is required, a wired interface will not suffice.

\subsubsection{\ac{SFP}}

This is a very common optical fibre interface, mainly used in network
switches. It can be very fast (10+ Gbit/s).

\url{https://www.snia.org/technology-communities/sff/specifications}

Breakout board

\url{https://sincsquared.com/sites/default/files/DataSheet_SS-SFP-SMA_0.pdf}

\url{https://osmocom.org/issues/3313}

\url{https://shop.trenz-electronic.de/en/TE0422-02-SFP-2-SMA-Adapter?path=Trenz_Electronic/Accessories/SFP/TE0422/REV02}

\ac{SFP} outputs differential signal so need a subtractor to change this to
a single ended signal appropriate for the laser. I believe these are known as
amplifiers.

See \url{https://electronics.stackexchange.com/questions/341592/differential-to-single-ended}

\url{http://blog.svenbrauch.de/2017/02/19/homemade-10-mbits-laser-optical-ethernet-transceiver/}

\subsubsection{Koruza}
This is an implementation of \ac{SFP}-based optical transmission system using
off-the-shelf \ac{SFP} modules operating at the infra-red light frequencies.

KORUZA - \url{https://github.com/IRNAS/KORUZA}

\ac{KAUST} already has some of these available to use. They claim up to
1 Gbit/s transfer rate. The nice thing about this project is that it is
all open source, both the hardware and the software.

There are currently no available \ac{SFP} transceivers working at the visible
light (350nm and 420nm).
