\section{Development - 07/04/2019}
The aim is firstly to study how the signalling works for the \ac{SFP}
modules. These are already available inside the KORUZA modules.

Two KORUZA systems have been dismantled and I have removed the following
components:

\paragraph{\textbf{KORUZA components removed}}
\begin{itemize}
\item{TP-Link MC220L}
\item{KORUZA \ac{SFP} breakout boards}
\item{KORUZA \ac{SFP} module}
\end{itemize}

As well as this, I also have the following:

\paragraph{\textbf{Other components used}}
\begin{itemize}
\item{Power Supply (used to power TP-Link boxes). Set to 5V, 1.4A maximum
(as each TP-Link box can draw around 0.6A)}
\item{Optical Fibre (for testing)}
\item{\ac{SFP} to Ethernet module (for testing)}
\item{Raspberry Pi (for testing)}
\item{Router}
\item{2 x Ethernet cables (between TP-Link boxes and Raspberry Pi / Router)}
\item{Oscilloscope to check signals}
\end{itemize}

The Raspberry Pi is set up as an Ethernet bridge to Wifi as described here:
\url{https://www.raspberrypi.org/documentation/configuration/wireless/access-point.md}.
In particular the section "\textbf{Using the Raspberry Pi as an access
point to share an internet connection (bridge)}".

At first I tried to point the \ac{SFP} modules at each other but the Link was
not stable. I then tried using an optical fibre but it would not plug in
properly. Instead of this, I managed to get two \ac{SFP} to Ethernet modules
and plugged then into each TP-Link box. This was enough to allow the Link
to be set up.

