\section{Development - 31/03/2019}
I have decided to pursue the 'Ethernet' approach, which is to use an \ac{SFP}
module, the same as all the other \ac{FSO} approaches do. The idea will be
to break out an \ac{SFP} module using a breakout board so that the laser
and photodiode can be attached. This should then be able to drive the system
much higher and provide a much faster link (gigabit).

\paragraph{\textbf{Components ordered}}
\begin{itemize}
\item{3 x \ac{SFP} to SMA Adapter
(\url{https://shop.trenz-electronic.de/en/TE0422-02-SFP-2-SMA-Adapter?path=Trenz_Electronic/Accessories/SFP/TE0422/REV02})}
\item{4 x SMA cables
(\url{https://www.thorlabs.com/thorproduct.cfm?partnumber=CA2912})}
\item{2 x Gigabit ethernet media converters - for the power supplies mainly.
(\url{https://www.amazon.com/TP-Link-Ethernet-Converter-Supporting-MC220L/dp/B003CFATL0})}
\end{itemize}

\section{Development - 03/04/2019}
The aim is firstly to study how the signalling works for the \ac{SFP}
modules. These are already available inside the KORUZA modules. The modules
were taken from the control laboratory, from Ibrahima
(\href{mailto:ibrahima.ndoye@kaust.edu.sa}{ibrahima.ndoye@kaust.edu.sa}).

Two KORUZA systems have been dismantled and I have removed the following
components:

\paragraph{\textbf{KORUZA components removed}}
\begin{itemize}
\item{TP-Link MC220L}
\item{KORUZA \ac{SFP} breakout boards}
\item{KORUZA \ac{SFP} module}
\end{itemize}

As well as this, I also have the following:

\paragraph{\textbf{Other components used}}
\begin{itemize}
\item{Power Supply (used to power TP-Link boxes). Set to 5V, 1.4A maximum
(as each TP-Link box can draw around 0.6A)}
\item{Optical Fibre (for testing)}
\item{\ac{SFP} to Ethernet module (for testing)}
\item{Raspberry Pi (for testing)}
\item{Router}
\item{2 x Ethernet cables (between TP-Link boxes and Raspberry Pi / Router)}
\item{Oscilloscope to check signals}
\end{itemize}

The Raspberry Pi is set up as an Ethernet bridge to Wifi as described here:
\url{https://www.raspberrypi.org/documentation/configuration/wireless/access-point.md}.
In particular the section "\textbf{Using the Raspberry Pi as an access
point to share an internet connection (bridge)}".

\section{Development - 07/04/2019}
Note that the Raspberry Pi WiFi does not seem to be stable. Was told by previous
student that it might require an external WiFi adapter in order to function
properly. Nevertheless we're going to throw it away anyway so doesn't matter.

At first I tried to point the \ac{SFP} modules at each other but the Link was
not stable. I then tried using an optical fibre but it would not plug in
properly. Instead of this, I managed to get two \ac{SFP} to Ethernet modules
and plugged then into each TP-Link box. This was enough to allow the Link
to be set up.

If the infra-red \ac{SFP} modules are aligned properly, the link is stable.
