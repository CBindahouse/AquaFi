\section{Conclusion}
This document has summarized underwater optical wireless network
topologies, their applications and related work in underwater wireless sensor
networks. Topologies have been proposed, and potential data rates and ranges
have been discussed.

Examining the applications for \ac{UOWN}s it becomes apparent that they will
be limited to small networks with only a few nodes, due to the reliability and
range issues. Thus the only applications for high data rates are video and
voice streaming. Another issue is that connectivity of \ac{UOWN}s must be
via the surface.

Considering this, the topologies for \ac{UOWN}s are limited to point to point
and star networks for simple deployments, and mesh networks for more complicated
deployments that might have multiple depth levels and longer range requirements.

It is much more interesting to consider optical networks as part of a hybrid
system where reliable but slow links utilise acoustic technology and high speed
but less reliable and shorter range links utilise optical technology. This can
even be considered when designing the \ac{MAC} protocol, for the Control and
Data planes respectively.