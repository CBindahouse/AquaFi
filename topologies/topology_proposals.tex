\section{Topology Proposals}
In order to propose topologies, we need to logically split the networks into
classes, based not just on the number of nodes but also their applications.

\begin{itemize}
\item{Point To Point}
\item{Star}
\item{Mesh}
\item{Hybrid}
\item{Daisy Chain}
\end{itemize}

\subsection{Topologies}

\subsubsection{Point to Point}
A Point to Point topology is only suitable for a single underwater node and a
node operating on the surface, or above. The range and data rate is limited
to that of one link.

\subsubsection{Star Topology}
A star topology is suitable for the following applications:
\begin{itemize}
\item{Short Distance - Optical links can only operate over short distances thus
we can only consider these for star topologies}
\item{Two Dimensional Architecture \cite{POMPILI2009778} - Where nodes are
anchored to the bottom of the seabed and connected to the surface via a single
vertical transceiver}
\end{itemize}

The range and data rate of a star network is limited to the range of one link.

In \cite{Diamant:2017:RUA:3205025.3205060} they claim the range can be
calculated based on the acoustic properties, although they do not give values
to the approximate ranges.

\subsubsection{Daisy Chain / Mesh Topology}
Daisy chain / mesh topology is suitable for the following applications
\begin{itemize}
\item{Long Distance - A mesh would increase redudancy compared to a simplistic
chain, but increase the routing complexity and thus time delay. Daisy chain is
a simpler version that has increased time delay due to the number of hops
but less routing complexity, with no redudancy.}
\item{Three Dimensional Architecture \cite{POMPILI2009778} - where nodes float
in mid water at different heights.}
\end{itemize}

The range of a mesh network can be considered as much longer than a point to
point or star network, however the range is not unlimited as these networks
are battery powered and cannot last indefinitely.

\subsubsection{Hybrid Topology}
A hybrid topology is suitable for the following applications
\begin{itemize}
\item{Three Dimensional Architctecture - A layered approach such as an inverted tree would be
suitable for a mutli-level network as described in
\cite{tier_based_underwater_routing}.}
\end{itemize}
