\section{Topology Proposals}
In order to propose topologies, we need to logically split the networks into
classes, based not just on the number of nodes but also their applications.

\begin{itemize}
\item{Point To Point}
\item{Star}
\item{Mesh}
\item{Hybrid}
\item{Daisy Chain}
\end{itemize}

\subsubsection{Point to Point}
A Point to Point topology is only suitable for a single underwater node and a
node operating on the surface, or above.

\subsubsection{Star Topology}
A star topology is suitable for the following applications:
\begin{itemize}
\item{Short Distance - Optical links can only operate over short distances thus
we can only consider these for star topologies}
\end{itemize}

\subsubsection{Daisy Chain Topology}
A daisy chain topology is suitable for the following applications
\begin{itemize}
\item{Long Distance - Long distance networks either require acoustic,
or a chain of optical nodes. RF would require to many nodes because of distance
limitations.}
\end{itemize}

\subsubsection{Mesh Topology}
A mesh topology is suitable for the following applications
\begin{itemize}
\item{Long Distance - A mesh would increase redudancy compared to a simplistic
chain, but increase the routing complexity and thus time delay.}
\end{itemize}

\subsubsection{Hybrid Topology}



