\section{Applications}
In order to examine underwater optical wireless network topologies
it is prudent to first examine the applications and resources that
we need to connect together.

\subsection{Point to Point}
Point to point is applicable in \ac{LOS} links, unless the beam has a large
beam divergence (although this will dramatically reduce the range and is
thus unpractical).

\subsection{Point to Multi-point}
Point to multi-point can only be realised by utilising \ac{NLOS} links, as
we can take advantage of scattering effects to communicate to multiple nodes.

\subsection{Above Surface to Underwater}
This includes nodes above the water such as on a satellite or \ac{UAV}.

\subsubsection{Relay Nodes}
This scenario is applicable for relay nodes if the above water node
can be considered 'static'. A relay node could communicate above water
using RF and underwater using an optical link.

\subsection{Surface to Underwater}
This includes nodes on the surface of the water such as on a ship
or a buoy, to nodues under the water such as \ac{ROV}s.

\subsection{Underwater to Underwater}
This is the main part of underwater wireless optical communications, and
includes divers, \ac{ROV}s, remote sensors and more.

\subsection{Static and Mobile Nodes}
As well as whether they are on the surface or under water, nodes can be
considered static (such as fixed sensors) or mobile (such as \ac{ROV}s).