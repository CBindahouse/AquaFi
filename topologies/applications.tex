\section{Applications}
In order to examine underwater optical wireless network topologies
it is prudent to first examine the applications and resources that
we need to connect together.

\subsection{Dimensionality}
As described in \cite{saeed2018underwater} underwater topologies can be broken
down into 1-dimensional, 2-dimensional and 3-dimensional architectures, and also
whether they are static or mobile. However, unless they are 1-dimensional
networks, which are always star networks, the classification of the
architectures depends on the application.

\subsection{Point to Point}
Point to point is applicable in \ac{LOS} links, unless the beam has a large
beam divergence (although this will dramatically reduce the range and is
thus unpractical).

\subsection{Point to Multi-point}
Point to multi-point can only be realised by utilising \ac{NLOS} links, as
we can take advantage of scattering effects from the surface or particulate
matter in the water to communicate to multiple nodes.

\subsection{Above Surface to Underwater}
This includes nodes above the water such as on a satellite or \ac{UAV}.

\subsubsection{Relay Nodes}
This scenario is applicable for relay nodes if the above water node
can be considered 'static'. A relay node could communicate above water
using RF and underwater using an optical link.

\subsection{Surface to Underwater}
This includes nodes on the surface of the water such as on a ship
or a buoy, to nodues under the water such as \ac{ROV}s.

\subsection{Underwater to Underwater}
This is the main part of underwater wireless optical communications, and
includes divers, \ac{ROV}s, remote sensors and more.

\subsubsection{Intra-Diver versus Inter-Diver}
Currently, \ac{RF} links are utilised between a diver's gas supply and
their computer, to monitor the pressure. Optical links would not be
a good replacement as there is no line of sight between the transmitter
and receiver, and the distances are low. However, inter-diver presents
a much more interesting approach, where optical links may be better suited
if the divers are located further away from each other. This has been
demonstrated as part of the Anglerfish \ac{UOWC} system, for example.

\subsection{Static and Mobile Nodes}
As well as whether they are on the surface or under water, nodes can be
considered static (such as fixed sensors) or mobile (such as \ac{ROV}s).

\subsection{Example Applications}
\begin{itemize}
\item{Ocean Sampling Networks}
\item{Environmental Monitoring}
\item{Undersea Exploration}
\item{Disaster Prevention}
\item{Command / Control}
\item{Assisted Navigation}
\item{Distributed tactical surveillance}
\item{Mine reconnaissance}
\item{Location and Positioning}
\end{itemize}

In \cite{iout_comprehensive} these are classified into 5 categories:
\begin{enumerate}
\item{Environmental Monitoring}
\item{Underwater Exploration}
\item{Disaster Prevention}
\item{Military}
\item{Others}
\end{enumerate}